\chapter{Extending surface hopping}
\label{chap:surface_hopping_ES}

\noindent FOB-SH is a variant of Tully's original fewest switches surface hopping \cite{FSSH_orig}. It has been used to simulate electron-nuclear dynamics in large systems of organic molecules and has been well tested against experimental studies and benchmarked against higher order studies \cite{FlickPolarons, Giannini2018Crossover, Giannini2019,             C9TC05270D, Carof2017FSSH, C9FD00046A, C9CP04770K, FOB-SH_Spencer,        C6FD00107F}. However, the code does not currently account for any electrostatic interactions. This presents a problem when looking at many systems; such as those with large amounts of disorder or those with polariseable molecules. 
%Need more motivation of implementing electrostatics in CP2K
\\\\
The standard Coulomb sum of partial charges is only conditionally convergent and extremely slow to calculate. The standard method for calculating electrostatic interactions is by decomposing interactions into long-range and short-range interactions (with corrections such as removing bonded terms etc\ldots). This is normally carried out with an Ewald sum \cite{Ewald} where a short-range interactions are calculated in real-space and long-range interactions are calculated in reciprocal-space. This results in 2 summations that are now unconditionally, quickly convergent. Further extensions to the standard Ewald technique provide an additional decrease in computational time by interpolating particles onto a grid and using fast fourier transforms to calculate the sums. In Wolf, 99 \cite{Wolf99}, a technique for removing the (expensive) reciprocal-space term from the sum altogether was proposed by ensuring charge neutrality within a cutoff sphere from each atom. This idea was developed to improve energy conservation and to remove discontinuities within the forces and energies \cite{Zahn2002, DSF}. In this work I will investigate the applicability of both the standard Ewald technique and a development of the Wolf method (named DSF \cite{DSF}) to calculate the electrostatic interactions within FOB-SH.
\section{Implementation details}
\subsection{Addition Subtraction Method}
\label{sect:addSubMethod}
\begin{figure}[ht]
  \includegraphics[width=\textwidth]{./img/ES/ForceEnerCalc.png}
  \caption{\label{fig:FE_Calc}A demonstration of the procedure to calculate diagonal elements of the Hamiltonian (site-energies). Red (blue) shapes represent a molecule in it's charged (neutral) state. A  horizontal line of these shapes represent the full system with all molecules; where a single molecule is in it's charged state. The arrow denotes which matrix element this saved as.}
\end{figure}
\noindent In FOB-SH, nuclear dynamics are determined by the Hamiltonian. The Hamiltonian is constructed such that the diagonal elements (site-energies) come from a classical forcefield and the off-diagonal elements (electron couplings) are proportional to the overlap of the diabatic wavefunctions. Each site-energy, $H_{\gamma \gamma}$, is defined as the potential energy of the system where the excess charge is localised on the molecule $\gamma$. For the avoidance of doubt, I will denote the molecule with the excess charge localised on it as the `charged' molecule and other molecules as `neutral'. The presence of the excess charge on molecule $\gamma$ results in different input parameters (such as the charge distribution or the length of bonds) than the other neutral molecules. This results in the calculation of site-energies and forces having to be repeated $N_{mol}$ times for each permutation of the charged molecule. This is summarised in figure \ref{fig:FE_Calc}. 
\\\\
To determine whether it is feasible to repeat the calculation of the electrostatic interactions $N_{mol}$ times a quick timing run was carried out. This simulated 250 pentacene molecules (9,000 atoms) and the time was measured to calculate the electrostatic interactions with the 3 methods already implemented within CP2K: Smooth Particle Mesh Ewald (SPME), Particle Mesh Ewald (PME) and standard Ewald. The measured time of a simulation without any electrostatics was then subtracted from each of these simulations to isolate the time spent on just the electrostatics. The results are given in figure  \ref{fig:ES_Timings}.
\begin{figure}[ht]
  \includegraphics[width=\textwidth]{./img/ES/InitialTimings.png}
  \caption{\label{fig:ES_Timings}The time taken to calculate just the electrostatic interactions within CP2K for a 9,000 atom system using various methods. PME is particle mesh Ewald, SPME is smooth-PME, Ewald is the standard ewald method. The dashed line shows the time taken for a single FOB-SH step.}
\end{figure}
\\
We can see that even a single calculation of the electrostatic interactions with the fastest method available within CP2K will take a comparable time to the rest of the surface hopping code. It is clear then that a more efficient method must be used to calculate the electrostatics in a more efficient way.
\\
\begin{figure}[ht]
  \includegraphics[width=\textwidth]{./img/ES/ForceEnerDecomp.png}
  \caption{\label{fig:enerF_decomp}A depiction of the decomposition of the forces and energies within FOB-SH. First the all neutral VDW forces/energies are computed (blue ovals). Second the intra-molecular forces for each charged (neutral) molecule, represented by a red (blue) rectangle. The site-energy/force is then computed as a summation of all molecules in their neutral state with a molecule in its neutral state subtracted and the same molecule in its charged state added.} 
\end{figure}
\\
Within the current FOB-SH implementation the forces and energies consist of intra-molecular components (bonds, bends, torsions etc\ldots) and inter-molecular components (Van der Waals forces provided by a Leonard-Jones potential). The same repetition of the calculation of forces and energies would, at first glance, be required for the correct calculation of these terms. However, an addition-subtraction scheme is used to reduce the calculation time from $O(N_{mol}, N_{atom}^2)$ to $O(N_{atom}^2)$. This is summarised in figure \ref{fig:enerF_decomp} and relies on the fact that the intra-molecular forces and energies can be decomposed into independent molecular contributions. In order to calculate the force on each atom and site-energy with molecule $\gamma$ in its charged state the code first calculates the force/energy with all molecules in their neutral state and then adds the contribution of molecule $\gamma$ in its charged state and subtracts the contribution of molecule $\gamma$ in its neutral state. We do not make the same adjustment for the VDW forces as the correction is negligible. This results in just 2 calculations of all forces and total energies rather than $O(N_{mol})$ calculations. Seeing as the electrostatics are normally one of the most expensive parts of a classical force-field it is particularly important that these are treated efficiently.
\\\\
The addition subtraction scheme applied to the electrostatic interactions cannot be the same as the one used for the intra-molecular interactions as separate molecules are not independent and energies and forces cannot be decomposed into molecular contributions for each different site-energy. However, a similar trick can be used to reduce the cost of the Ewald sum. In the following work, the recalculation method references the method of looping over all molecules and recalculating energies and forces without optimisations. The addition-subtraction scheme is explained in the proceeding chapters.
\subsection{Ewald Equations and the additional subtraction scheme}
The standard Ewald summation for evaluating electrostatic energies in molecular dynamics simulation are given below:
\begin{eqnarray}
  \begin{split}
E_{coul}\left(\mathbf{r}^{N}\right)
=
% Real Space Sum
\frac{1}{2}\frac{1}{4 \pi \epsilon_0} \sum_{\mathbf{n}} \sum_{j}^{N_{at}} \sum_{i}^{N_{at}} q_i q_j \frac{erfc\left( \alpha \cdot |\mathbf{r}_{ij} + \mathbf{n}|\right)}{|\mathbf{r}_{ij} + \mathbf{n}|} \Theta\left( r_{cut} - |\mathbf{r}_{ij} + \mathbf{n}| \right)
\\
% Reciprocal Space Sum
+
\frac{1}{2\pi V} \ \frac{1}{4 \pi \epsilon_0} \sum_{\mathbf{k} \neq 0} \frac{1}{|\mathbf{k}|^2} e^{-\frac{\pi^2 \ |\mathbf{k}|^2}{\alpha^2}} \ \left|\sum_{j}^{N_{at}} q_{j} e^{2\pi i \mathbf{k} \cdot \mathbf{R}_{j}}\right|^2 \\
% Self Term
- \frac{\alpha}{\sqrt{\pi}} \frac{1}{4 \pi \epsilon_{0}} \sum_{j} q_{j}^2
\\
% Bonded Correction Term
- \frac{1}{2} \frac{1}{4 \pi \epsilon_0} \sum_{j}^{N_{at}} \sum_{i}^{N_{at}} q_i q_j \frac{erfc\left( \alpha \cdot |\mathbf{r}_{ij}|\right)}{|\mathbf{r}_{ij}|} \Theta\left( r_{cut} - |\mathbf{r}_{ij}| \right) 
	\end{split}
\label{eq:EwaldStd}
\end{eqnarray}
In equation \eqref{eq:EwaldStd}, the first term is the real-space sum. This sums over all periodic images ($\mathbf{n}$) and pairs of atoms $i$, $j$ within a cutoff imposed by the Heaviside step function $\Theta(r_{cut} - |\mathbf{r}_{ij}+\mathbf{n}|)$. The distance between atoms is given by $\mathbf{r}_{ij} = \mathbf{r}_{i} - \mathbf{r}_{j}$, the charge on atom $i$ is given by $q_{i}$ and alpha is a convergence parameter. The factor $\frac{1}{2}$ accounts for any double counting of atoms. The second term is the most expensive part of this calculation and sums over reciprocal-space vectors $\mathbf{k}$ and atoms, $j$. $\mathbf{R}_{j}$ represents the position vector of atom $j$. The third term is the constant self-energy term and the fourth corrects for bonded  (intra-molecular) interactions. The bonded interactions may be ignored in the real-space sum, this correction removes their effect from the reciprocal-space sum. As these 4 summations are independent we can look at each one separately when implementing the addition-subtraction scheme, starting with the simplest -the self-energy term. Note in this section I will only discuss the energies, the forces are very similar and their equations are given in appendix \ref{ap:Ewald}.
\subsection{Self-energy addition subtraction scheme}
For each site-energy, $\gamma$, we must recalculate the full forces and energies with the excess charge located on molecule $\gamma$. This is demonstrated in equation \eqref{eq:SelfNoScheme}. Note that for brevity I have replaced the factor $\frac{1}{4 \pi \epsilon_0}$ with $\eta$.
\begin{equation}
  E_{self}^{\gamma} = \frac{\alpha}{\sqrt{\pi}} \eta \left[\sum_{j \not\in \gamma} \left(q^{n}_{j}\right)^2  + \sum_{j \in \gamma} \left(q^{c}_{j}\right)^2\right]
  \label{eq:SelfNoScheme}
\end{equation}
In the above equation the Ewald self-energy correction contribution for site-energy $\gamma$ is simply a sum of squared charges of molecules in their neutral state, for atoms belonging to molecules that aren't $\gamma$ plus the sum of squared charges of atoms within $\gamma$. The molecule $\gamma$ will have a different values due to the presence of the excess charge carrier. This is represented by the superscript $n$ and $c$ where $q^{n}_{j}$ represents the charge on atom $j$ where the force-field for the molecule it belongs to has been parameterised in it's neutral state (i.e. without an excess charge carrier). $q^{c}_{j}$ represents the charge on atom $j$, where the force-field for the molecule the atom belongs to has been parameterised with an excess charge carrier localised on it (in its charged state).
\\\\
For a single molecule system this value is the same for all $\gamma$ and no optimisations are required, except to calculate this value once and use it for each $\gamma$. However, for a more complex system the addition subtraction scheme used is given in equation \eqref{eq:SelfScheme}.
\begin{equation}
  E_{self}^{\gamma} = \eta \underbrace{\frac{\alpha}{\sqrt{\pi}} \sum_{j}^{N_{at}}\left(q_{j}^{n}\right)^2}_{\text{Calculated Once}} + \underbrace{\frac{\alpha}{\sqrt{\pi}} \eta \sum_{j \in \gamma} \left[ (q^{c}_{j})^2 - (q^{n}_{j})^2 \right]}_{\text{Calculated for each $\gamma$}}
  \label{eq:SelfScheme}
\end{equation}
In equation \eqref{eq:SelfScheme} we have removed the $\gamma$ index from the most expensive part of the sum; this means we can calculate it once and store it. In the second term we only sum over atoms in charged molecule $\gamma$ and remove the contribution from molecule $\gamma$ in its neutral state and add the contribution from molecule $\gamma$ in its charged state. Seeing as the correction part of equation \eqref{eq:SelfScheme} is the only part repeated from each $\gamma$ this reduces the cost of this calculation from $O(N_{mol}, N_{atom})$ to just $O(N_{atom})$. The same idea is used for the remaining terms in the Ewald sum.
\subsection{Real-space addition subtraction}
The real-space term is more complicated than the self-energy term, though the idea is the same. That is, the fully neutral contribution is calculated and for individual sites/molecules a correction is applied. This is shown in equation \eqref{eq:RealScheme}
\begin{eqnarray}
  \begin{split}
	  E^{\gamma}_{real} &= \frac{\eta}{2} \sum_{\mathbf{n}} \sum_{j}^{N_{at}} \sum_{i}^{N_{at}} q^{n}_i q^{n}_j  \ R^{dir}(\mathbf{r}_{ij} + \mathbf{n}) \\
	  &+ \frac{\eta}{2}\sum_{j \in \gamma, i \in \gamma} (q_j^c q_i^c - q_j^n q_i^n) \ R^{dir}(\mathbf{r}_{ij} + \mathbf{n}) \\
	  &+ \frac{\eta}{2}\sum_{j \in \gamma, i \not\in \gamma} (q_j^c - q_j^n)q_i^n \ R^{dir}(\mathbf{r}_{ij} + \mathbf{n}) 
  \end{split}
  \label{eq:RealScheme}
\end{eqnarray}
In equation \eqref{eq:RealScheme} the most expensive summation ($O(N_{atom}^2)$) is the first term. Fortunately, we can once again calculate this once and use the same value for each site-energy. This first term calculates all interactions between atoms belonging to molecules in their neutral state (neutral-neutral interactions). The next two terms show the addition-subtraction correction. The second term shows a sum over all pairs of atoms in the charged molecule, $\gamma$. In this term we subtract any neutral-neutral interactions and replace them with any charged-charged interactions. This scales as $O(N_{\text{atom per mol}})$ and is repeated $N_{mol}$ times so the full correction scales as $O(N_{atom})$. The third term replaces any interactions of atoms on the charged molecule with its environment (neutral molecules), hence it removes neutral-neutral interactions and replaces them with charged-neutral interactions. This scales as $O(N_{\text{atom per mol}}, N_{atom})$ and is repeated $N_{mol}$ times, resulting in an ultimate scaling of $O(N_{atom}^2)$. Therefore, this optimisation scales in the same manner as a single calculation of the Ewald interactions and any additional overheads will be minimal. For the avoidance of doubt, in equation \eqref{eq:RealScheme} I have replaced the complementary error function and Heaviside step function in equation \eqref{eq:EwaldStd} with the term $R^{dir}(\mathbf{r}_{ij} + \mathbf{n})$.
\subsection{Bonded corrections addition subtraction}
The bonded correction terms remove electrostatic contributions to energies (and forces) for atoms that are bonded. This is because interactions are already accounted for by the intra-molecular force-field (bonds, bends, torsions etc\ldots). These interactions occur within molecules and their contribution can be decomposed into molecular contributions. These interactions can therefore be handled in the same way as the intra-molecular addition-subtraction scheme as discussed in section \ref{sect:addSubMethod}.
\subsection{Reciprocal-space addition subtraction}
The reciprocal energies can be optimised using the addition-subtraction technique. However, the forces cannot. This is a big problem for any implementation of Ewald electrostatics within surface hopping as the electrostatic part of the Ewald sum is by far the most expensive. In fact in the same 250 molecule system as in figure \ref{fig:ES_Timings} the reciprocal-space component took 88\% of the calculation time. In larger systems this increases. Repeating this calculation $N_{mol}$ times would be far too slow and would limit the surface hopping code to small systems of tens of molecules. However, the damped shifted forces technique (DSF) \cite{DSF} can be used to approximate the electrostatic interactions without the reciprocal force term. For completeness I have included the addition-subtraction scheme for the reciprocal-space energies below in equation \eqref{eq:RecipScheme} and for the forces in appendix \ref{ap:Ewald}.
\begin{equation}
  E^{\gamma}_{recip} = \frac{1}{2 \pi V} \sum_{\mathbf{k} \neq 0} \frac{1}{|\mathbf{k}|^2} e^{\frac{\pi^2 |\mathbf{k}|^2}{\eta^2}} \left| \sum_{j}^{N_{at}} q^{n}_{j} e^{2 \pi \mathbf{k} \cdot \mathbf{R}_{j}}  + \sum_{j \in \gamma}^{N_{at}} (q^{c}_{j} - q^n_j) e^{2 \pi \mathbf{k} \cdot \mathbf{R}_{j}} \right| ^2
  \label{eq:RecipScheme}
\end{equation}
Once again in equation \eqref{eq:RecipScheme} the summation over all atoms can be calculated once and reused for each site-energy $\gamma$. This calculates all neutral-neutral interactions. The additional sum over atoms belonging to molecule $\gamma$ is then repeated $N_{mol}$ times for each site-energy $\gamma$.
\section{Timing the electrostatics implementation}
\begin{figure}[ht]
  \includegraphics[width=\textwidth]{./img/ES/TimingsReCalc_vs_AddSub.png}
  \caption{\label{fig:AddSubTimings}Time taken to run surface hopping and electrostatics for various lengths of 1D Ethylene chain (left) and 800 molecule pentacene plane (right). Darker colors show data from the recalculation method for the electrostatics and less saturated colors to the right show data from the addition subtraction scheme. Green bars show the time taken to calculate real-space interactions, red is reciprocal, yellow is the bonded corrections and blue shows all other parts of the surface hopping code. In the right pane reciprocal interactions are omitted as they took too long to run.}
\end{figure}
In figure \ref{fig:AddSubTimings} we can see the time taken for a single step of a surface hopping simulation for various lengths of a 1D Ethylene chain. We see as the chain size increases it becomes more important that electrostatic interactions are efficiently handled. For the 800 molecule pentacene plane, the reciprocal calculations were taking far too long and had to be turned off to measure the time taken for the other components. We see also that even with the addition-subtraction scheme the full reciprocal-space calculations don't see a very significant speedup. This is because the calculation of the forces are still repeated $N_{mol}$ times as they cannot be optimised in the same way. A small speedup is seen due to the addition-subtraction scheme being used with the reciprocal-space energies. However, we see that the addition-subtraction scheme offers a major speedup for all other components. It is also vital that the results outputted are correct. I have tested both the recalculation method and the addition-subtraction method against standard CP2K calculations to ensure the implementation is correct.

\subsection{Testing the electrostatics implementation}
To test the recalculation of site-energies and forces within CP2K a 10 molecule Ethylene chain was used. In order to calculate the site-energies and forces with standard MD in CP2K, 10 different simulations were carried out. In each one of these simulations a different molecule was chosen to have charged geometry and the rest were chosen to have neutral geometry in the input files. A single step of MD was then carried out and forces and energies were outputted. These forces and energies were subsequently compared to the forces and energies outputted by the recalculation method and the results are shown in figure \ref{fig:ReCalcTest}
\begin{figure}[ht]
  \includegraphics[width=\textwidth]{./img/ES/10_mol_FIST.png}
  \caption{\label{fig:ReCalcTest}A comparison of forces and energies calculated with the recalculation method and the forces and energies calculated through multiple CP2K calculations.}
\end{figure}
We see in figure \ref{fig:ReCalcTest} that the values of energy and forces as calculated with CP2K's standard MD package (FIST) and my implementation of a simple loop carrying out the same thing in the code are exactly the same. In fact the maximum absolute difference between results was 5$\times 10^{-13}$ i.e. numerical error. This confirms the implementation of the recalculation scheme. The reason only 5 values for the site-energies can be seen is some points are extremely close and it is hard to see differences in them. This shows the recalculation method can now be used to benchmark the addition-subtraction scheme.
