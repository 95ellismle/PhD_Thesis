\chapter{Extending surface hopping for larger systems}
\label{chap:surface_hopping_ext}
Fragment-orbital based surface hopping (FOB-SH) is a technique developed within the Blumberger group \cite{spencer_fob-sh:_2016} designed to simulate large molecular systems. It has had much success in the study of organic crystalline materials \cite{giannini_crossover_2018, carof_detailed_2017}. Most notably the electron/hole mobilities of a variety of common organic semi-conducting materials were measured within a factor of 2 of experimental measurements. However, in order to study very large amorphous and semi-crystalline systems some more memory/computation optimisations were required and electrostatic interactions (which weren't important in previous systems) needed to be accounted for. In this chapter I outline some minor improvements I implemented within the surface hopping code as well as the method used to implement the electrostatic interactions.

\section{Code Optimisations}
