\chapter{General Conclusions and Outlook}
\label{chap:outlook}
Although nonadiabatic chemistry dictates many of life's vital processes, such as: photosynthesis; electron transfer; and light detection \cite{Dral2018}, the complexity of the underlying physics has meant simulating such systems is relatively novel. Electron transfer has been of particular interest in this work, specifically the transport of electrons or electron-holes within organic semiconductors. In this field, a useful quantity of merit is the charge carrier mobility. That is how quickly charge moves within a system as a response to an electric field. In organic semiconductors, the most common tool for simulating charge carrier mobilities is the Marcus's master rate equation (typically used coupled with a Monte-Carlo method). The validity of the Marcus rate in particular is often disputed due to the large quantum delocalisation of the electron wavefunction within the OS systems. However, the quantum delocalisation isn't sufficient to warrant band theory calculations either. To fill this gap, nonadiabatic molecular dynamics techniques such as surface hopping, Ehrenfest and CTMQC have been developed. This work has been concerned with the further development of these techniques and their application to simulations of charge transfer.
\\\\
In the first 2 chapters my implementation of the new CTMQC technique was discussed. However, there is still some way to go before it's widespread adoption as a alternative to surface hopping. CTMQC's main advantage is that it has been derived from first principles and claims to rigorously handle the overcoherence problem that has hampered surface hopping. While it is true that CTMQC performs very well on simple 1D Tully model systems, even reproducing exact quantum dynamics correctly (given correct parameterisation), it is let down by 2 key problems. First, is it's instability when applied to even moderately complex molecular systems. Second, is the necessity to parameterise the quantum momentum to achieve accurate decoherence. In order to solve these problems a new way to construct the nuclear density may be required, leading to a new equation for calculating the quantum momentum. Perhaps further studies on 2/3D generalisations of the Tully models or single molecular systems may also help shed light on the algorithm. Finally, in order to be a true rival of the surface hopping technique, effort must be spent on finding efficiencies in it's implementation. I have presented the CTMQC equations within the FOB-SH framework, where diabatic expansion coefficients are propagated rather than their adiabatic counterparts. However, 2 key quantities: the quantum momentum and the adiabatic momentum terms, still require the calculation of adiabatic quantities such as the spatial derivative of the adiabatic energies which leads to a major increase in computational time. Perhaps a re-derivation in the diabatic basis would lead to increased performance and maybe even more stable propagation.
\\\\
Another nonadiabatic molecular dynamics (NAMD) technique studied in this work is fewest switches surface hopping, namely fragment orbital-based surface hopping (FOB-SH). This is a tried and tested method developed by multiple past and present members of the group and is able to simulate large systems, often within experimental accuracy. In this work I've applied FOB-SH to an disordered pentacene systems, created by a melt-quench technique with various quench times. This study is one of the first (to the author's knowledge) to simulate structures with such a wide spectrum of disorder. This presented a unique opportunity to showcase FOB-SH's ability and to compare hole-mobilities spanning several orders of magnitude to experiment. At both ends of the spectrum, experimental mobilities agreed well with the simulated results. However, the amorphous values were on the high end of a wide spread of experimental results. This is thought to be due to the lack of electrostatic interactions within FOB-SH. While this isn't thought to have a significant effect on hole mobilities within highly ordered crystals it will within much more disordered amorphous systems and at crystal interfaces. For this reason, I have implemented and tested 2 electrostatics methods within FOB-SH. The first, full Ewald, is a representation of the coulomb sum, with long range interactions calculated in reciprocal space and short range interactions calculated in real space. The second, DSF, is an approximation to Ewald electrostatics derived by simply omitting the long range interactions and ensuring net charge-neutrality within a cutoff sphere with the use of virtual charges. I found that, although the Ewald sum is exact, it is far too expensive to use in realistic simulations and DSF must be used instead. Though, not thoroughly tested, initial results are very promising and pave the way for a new class of materials to be studied.
