\maketitle
\makedeclaration

\begin{abstract} % 300 word limit
	This work will be split into \replace{2}{two} parts. The first will be concerned with the implementation of a novel nonadiabatic molecular dynamics technique, derived as the semi-classical limit of exact factorisation, named Coupled-Trajectory Mixed Quantum-Classical Molecular Dynamics (CTMQC). I will investigate its current formulation within the literature, highlight some current pitfalls ---and suggest ways to alleviate them--- and present results of my implementation. I will also give results of the integration of this technique within the fragment-orbital based framework (FOB). This is designed to allow the fast calculation of electronic couplings through formulating the equations in a diabatic basis. Initially, the CTMQC algorithm will be applied to the 1D Tully toy models and later to an Ethylene dimer. We will see that, although the Tully model results are very promising, instabilities in the calculation of key quantities makes the current algorithm unusable for molecular systems.
\\\\
The second part will be concerned with a well-tested, semi-classical technique, based on Tully's fewest switches surface hopping (FOB-SH). I will apply this to nanoscale systems of pentacene; will investigate the effect a variable quench time in a melt-quench scheme has on the crystallinity of these pentacene systems; and discuss how the resulting nanostructure affects charge transport dynamics. Finally, I will also discuss my implementation of 
\replace{2}{two} methods to calculate electrostatic interactions within FOB-SH. I will show how an addition-subtraction scheme and the damped shifted forces method (an approximation to full Ewald) can be used to optimise the calculations and test these methods against full Ewald electrostatics.
\end{abstract}
\clearpage
\chapter*{Impact Statement}
Many processes critical to life and essential technologies such as photosynthesis, vision and charge transport, are nonadiabatic. That is, a system in which electronic states can change energy level nonradiatively. However, these processes often occur at very short timescales (fs-ns) and/or at very small lengthscales (nm-$\mu$m). Therefore, they have until quite recently, been difficult to study experimentally. To compound matters, the systems of interest are often quantum mechanical in nature, making large systems (beyond nm or ns) expensive to study computationally/theoretically. This often makes comparison of computational results to experiment difficult and it is only with carefully chosen approximations to accepted theory (such as Schr\"odinger's equation) that these systems can be modelled. Perhaps the most crucial approximation is the semi-classical approximation, i.e. treating as much of the system with classical mechanics as is reasonable. This is often achieved by treating the nuclear degrees of freedom as classical point particles and the electronic degrees of freedom quantum mechanically. In this work, I will investigate 
\replace{2}{two} techniques that make use of this approximation: \replace{the current, most widespread}{a popular} nonadiabatic molecular dynamics technique, Tully's fewest switches surface hopping (FSSH); and a \add{lesser known} newcomer to the field, coupled-trajectory mixed quantum-classical molecular dynamics (CTMQC).
\\\\
This work \replace{will be}{is}, to my knowledge, the first to present the pitfalls of the CTMQC technique, and partially address them. I will show tests of my implementation, verifying that it works, before presenting the cause of the problems. This is important as CTMQC shows some extremely promising results with simple model systems, especially when it comes to the correct account of quantum coherence -a shortcoming that has troubled users of the more conventional FSSH. I will also present an implementation of CTMQC within the extremely efficient fragment orbital based framework. This framework has been used previously with surface hopping to simulate systems of thousands of molecules -helping to bridge the gap between experimentally accessible length/timescales and computationally accessible ones. This is vital to computational models, which should be validated against experiment.
\\\\
In addition to aiding in the development of theory, this work will apply a technique named fragment orbital based surface hopping (FOB-SH) to model charge transport in systems of varying levels of amorphicity/crystallinity. It is in practise extremely difficult to produce highly pure, organic, single crystals that lack defects. This is especially true in the mass production of consumer electronics. For this reason, it is important to be able to model systems with high levels of disorder, to help guide experiment and the production of new (organic semiconductor) technologies. This work shows that the FOB-SH method can be accurately applied to these systems and retrieve experimentally comparable results spanning several orders of magnitude.
\\\\
Finally, a vital property in the simulation of many materials -electrostatic interactions- is investigated. In particular, the implementation of electrostatic calculations within FOB-SH. Currently, these are inefficiently implemented making it infeasible to apply them to systems larger than tens of molecules. I will present \replace{2}{two} implementations of electrostatics within FOB-SH that will allow the simulation of systems of hundreds to thousands of organic molecules.  This will pave the way for new classes of materials to be studied nonadiabatically with FOB-SH.


\begin{acknowledgements}
I'd like to start by thanking all the members of the Blumberger group for their support, I know without them I wouldn't have completed the PhD. I would especially like to thank Prof Antoine Carof for holding my hand through the first several months, and Dr Samuele Giannini for the rest! Chris Ahart, Dr Xiuyun Jiang, Dr Orestis Ziogos, Dr Xioajing Wu, Jan Elsner, Prof Soumya Ghosh, Dr Patrick Gutlein, Dr Hui Yang, Dr Lijljana Stojanovic, Dr Roohollah Hafizi, Dr Wei-Tao Peng, Daniel Holub and Prof Zdenek Futera also all deserve a special mention. I, of course, would especially like to thank my supervisor Jochen Blumberger for all his guidance and many useful discussions throughout my PhD.
\\\\
I would also like to thank all the friends and family who have supported me. Though, special thanks should go to: my parents, grandparents and brother; my partner's family ---especially for tolerating  living with me for several months during lockdown!; Victor and Rhiannon.
\\\\
Most of all, I would like to acknowledge my partner Liam for all his help and support throughout the PhD, and I would like to dedicate this thesis to him.
\end{acknowledgements}

\setcounter{tocdepth}{2} 
% Setting this higher means you get contents entries for
%  more minor section headers.

\tableofcontents
\listoffigures
\listoftables

