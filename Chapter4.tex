\chapter{Extending surface hopping}
\label{chap:surface_hopping_ext}
Fragment-orbital based surface hopping (FOB-SH) is a technique developed within the Blumberger group \cite{spencer_fob-sh:_2016} designed to simulate large molecular systems. It has had much success in the study of organic crystalline materials \cite{giannini_crossover_2018, carof_detailed_2017}. Most notably the electron/hole mobilities of a variety of common organic semi-conducting materials were measured within a factor of 2 of experimental measurements. However, in order to study very large amorphous and semi-crystalline systems some more memory/computation optimisations were required and electrostatic interactions (which weren't important in previous systems) needed to be accounted for. In this chapter I outline some minor improvements I implemented within the surface hopping code as well as the method used to implement the electrostatic interactions.

\section{Code Optimisations}

\section{Electrostatics interaction within FOB-SH}
FOB-SH is a variant of Tully's original fewest switches surface hopping \cite{} (ref John Tully 1990 original surface hopping paper). The electron and nuclear dynamics are dictated by the Hamiltonian, where the spatial derivative of the diagonal elements (site-energies) give the nuclear forces which come from a classical forcefield and the off-diagonal elements (electron couplings) are proportional to the overlap of the diabatic wavefunctions. Each site-energy, $H_{\gamma \gamma}$, is defined as the potential energy of the system where the excess charge is localised on a single molecule, $\gamma$. The input parameters such as the charge distribution or the strength of the bonds are different for molecule $\gamma$ than the other molecules, resulting in different forces/potentials. The different potentials from each permutation of the charge state are saved as the site-energies, the forces are saved in a different array of size $(N_{states}, N_{atoms}, 3)$. To implement electrostatic interactions the potential and the forces within the surface hopping framework will have to be altered.
\\\\
To calculate the electrostatic contribution from each atom to the forces and potential the standard Ewald method was chosen, with some additional tricks to reduce the computational cost. The Ewald method, first published by Paul Ewald in 1921 (NEED REF), is a near ubiquitous method used to evaluate the potential and forces from a set of point charges. This is equivalent to carrying out the much slower to converge (NEED REF) and conditionally convergent traditional coulomb sum. The Ewald method relies on separating the traditional coulomb summation into 2 quickly converging parts; a long range term and short range term. The short range term is carried out in real space and the long range term is carried out in carried out in reciprocal space
