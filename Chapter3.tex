\chapter{CTMQC applied to molecular systems}
\label{chap:molecular_systems}
In order to apply CTMQC to large molecular systems (hundreds of molecules) a new way to construct the Hamiltonian is needed. In this work I have implemented the CTMQC equations within the fragment-orbital based framework. This relies on the equations being expressed in a pseudo-diabatic basis and the Hamiltonian being constructed in 2 parts: the diagonal elements (site energies) and the off-diagonals (electronic couplings). The basis is termed 'pseudo-diabatic' due to the fact that non-adiabatic coupling vectors are small but non vanishing, this results in a basis where the excess charge carrier is strongly but not strictly localised on a single molecule. Within the Hamiltonian, the site energies are calculated via classical force-fields and the electronic couplings are calculated via the analytic overlap method \cite{gajdos_ultrafast_2014, spencer_fob-sh:_2016} (AOM). In this method the coupling elements are assumed to be proportional to the overlap between the highest singly occupied molecular orbitals (SOMO) on the donor and acceptor molecules (see equation \eqref{eq:AOM_proportional}). This approximation is often used in the literature, e.g. in the fragment orbital density functional theory \cite{KirkpatrickJ2008,C2CP41348E,TroisiA2002} (FODFT) method and has been shown to be valid for $\pi$-conjugated molecules \cite{KubasA2014, gajdos_ultrafast_2014}.
\begin{equation}
  H_{ab} = C \langle \varphi_{a} | \varphi_{b} \rangle = C S_{ab}
  \label{eq:AOM_proportional}
\end{equation}
Where $\varphi_{a(b)}$ represents a singly occupied molecular orbital on the donor (acceptor) and C is the scaling constant and comes from DFT parameterisation. The singly occupied molecular orbitals are calculated as a linear combination of Slater-type orbitals (STO) as in equation  \eqref{eq:SOMO_def}. In this equation we loop over each molecule in the atom and sum the size of the contribution, $c_{p\pi, i}$, multiplied by the STO, $p_{\pi, j}$. In this case the STO is represented by a p-orbital. The size of the p-orbital on each atom, $c_{p\pi, i}$, is parameterised before the simulation with DFT.
\begin{equation}
  | \varphi_{mol} \rangle = \sum_{i \in mol}^{N_{\text{atoms}}} c_{p\pi, i} | p_{\pi, i} \rangle
  \label{eq:SOMO_def}
\end{equation}


\section{Basis Transformation}
In order to use the FOB method, as stated above the CTMQC equations in the adiabatic basis must be transformed to the diabatic basis. In the following derivation $C_{l}$ will represent the adiabatic expansion coefficient corresponding to state l and $u_{l}$ will represent the orthogonal diabatic expansion coefficients.
\\\\
The CTMQC equations in the adiabatic basis are given below in equation \eqref{eq:adForce} of the forces and \eqref{eq:adCoeff} coefficients:
\begin{align}
  \begin{split}
	  \mathbf{F}_{\nu}^{(I)} = &- \sum_{k} |C_{k}^{(I)}|^2 \nabla_{\nu}E_{k}^{(I)} - \sum_{k, l} C_{l}^{* (I)} C_{k}^{(I)} \left(E_{k}^{(I)} - E_{l}^{(I)} \right) \mathbf{d}_{\nu, lk}^{ad, (I)} \\
	  &- \sum_{l,k} |C_{l}^{(I)}|^2 \left( \sum_{\nu'}^{N_{n}} \frac{2}{\hbar M_{\nu'}} \mathcal{Q}_{\nu', lk}^{(I)} \cdot \mathbf{f}_{l, \nu'}^{(I)} \right)\left[ \mathbf{f}_{k, \nu}^{(I)} - \mathbf{f}_{l, \nu}^{(I)} \right] |C_{k}^{(I)}|^2 
  \end{split}
  \label{eq:adForce}
\end{align}

\begin{align}
  \begin{split}
	\dot{C}_{l}^{(I)} = &-\frac{i}{\hbar} E_{l}^{(I)} C_{l}^{(I)} - \sum_{k} C_{k}^{(I)} \sum_{\nu=1}^{N_{n}} \frac{\mathbf{P}_{\nu}^{(I)}}{M_{\nu}} \cdot \mathbf{d}_{\nu, lk}^{ad, (I)} \\
	&- \sum_{\nu=1}^{N_{n}} \sum_{k} \frac{\mathcal{Q}_{\nu, lk}^{(I)}}{\hbar M_{\nu'}} \cdot \left[\mathbf{f}_{k, \nu}^{(I)} - \mathbf{f}_{l, \nu}^{(I)} \right] |C_{k}^{(I)}|^2 C_{l}^{(I)}
  \end{split}
  \label{eq:adCoeff}
\end{align}

\noindent Where:
\begin{itemize}
  \item $E_{k}^{(I)}$ is the adiabatic energy for state k and trajectory I
  \item $C_{k}^{(I)}$ is the adiabatic expansion coefficient for state k and trajectory I
  \item $\mathbf{P}_{\nu}^{(I)}$ is the classical momentum of atom $\nu$ on trajectory $I$
  \item $\mathbf{d}_{\nu, lk}^{ad, (I)}$ is the nonadiabatic coupling vector (given in the adiabatic basis)
  \item $M_{\nu}$ is the mass of nuclei $\nu$
  \item $\mathcal{Q}_{\nu, lk}^{(I)}$ is the quantum momentum vector for atom $\nu$ corresponding to the $lk$ pair of states in trajectory $I$
  \item $\mathbf{f}_{l, \nu}^{(I)}$ is the adiabatic momentum on state l, atom $\nu$ and trajectory $I$
\end{itemize}

\subsection{Coefficients}
To transform the equation for the propagation of the coefficients it is far neater to use the notation of linear algebra as in equation \eqref{eq:LA_Coeff} below:
\begin{equation}
	\dot{\mathbf{C}}^{(I)} = \mathbb{X}_{\nu}^{(I)} \mathbf{C}^{(I)} = \left(\mathbb{X}_{eh, \nu}^{(I)} + \mathbb{X}_{qm, \nu}^{(I)}\right) \mathbf{C}^{(I)}
	\label{eq:LA_Coeff}
\end{equation}
Where the $\mathbb{X}$ matrices are defined as in equations \eqref{eq:Xeh_def} and \eqref{eq:Xqm_def} below.
\begin{equation}
  \mathbb{X}_{lk, \nu}^{eh (I)} = -\frac{i}{\hbar} E_{l}^{(I)} - \sum_{\nu}^{N_{n}}\frac{\mathbf{P}_{\nu}^{(I)}}{M_{\nu}} \cdot d_{lk, \nu}^{ad, (I)}
  \label{eq:Xeh_def}
\end{equation}

\begin{equation}
  \mathbb{X}_{ll, \nu}^{qm (I)} = -\sum_{\nu=1}^{N_{n}} \sum_{k} \frac{\mathcal{Q}_{\nu, lk}^{(I)}}{\hbar M_{\nu'}} \cdot \left[ \mathbf{f}_{k, \nu}^{(I)} - \mathbf{f}_{l, \nu}^{(I)} \right]  |C_{k}^{(I)}|^2
  \label{eq:Xqm_def}
\end{equation}


\noindent Using the identities:
\begin{align}
  \mathbb{U}^{-1} &= \mathbb{U}^{\dagger} \\
  \mathbf{C}^{(I)} &= \mathbb{U}^{\dagger (I)} \mathbf{u}^{(I)} \\
  \dot{\mathbf{C}}^{(I)} &= \dot{\mathbb{U}^{\dagger (I)}} \mathbf{u}^{(I)} + \mathbb{U}^{\dagger (I)}\dot{\mathbf{u}}^{(I)}
\end{align}
Where $\mathbb{U}^{(I)} = \langle \phi_{l}^{(I)} | \psi_{n}^{(I)} \rangle$ is the unitary transformation matrix transforming from the diabatic to adiabatic basis. The $\mathbf{u}^{(I)}$ terms are the diabatic expansion coefficients on trajectory I.
\\\\
\noindent After some algebra we arrive at:
\begin{equation}
  \dot{\mathbf{u}}^{(I)} = \underbrace{\left(\mathbb{U}^{(I)} \mathbb{X}_{eh} \mathbb{U}^{\dagger (I)} + \mathbb{U}^{(I)}\mathbb{U}^{\dagger (I)}\right) \mathbf{u}^{(I)}}_{\text{Ehrenfest}} + \underbrace{\left(\mathbb{U}^{(I)} \mathbb{X}_{qm} \mathbb{U}^{\dagger (I)} \right) \mathbf{u}^{(I)}}_{\text{Quantum Momentum}}
  \label{eq:diabMatEq}
\end{equation}

\noindent In equation \eqref{eq:diabMatEq} I've separated the contribution from Ehrenfest and the contribution from the new quantum momentum terms. The Ehrenfest part can be shown to reduce to a simpler form (see Spencer, 2016 \cite{spencer_fob-sh:_2016} and Carof, 17 \cite{carof_detailed_2017} for more details). The quantum momentum term must be coded up as shown -with the transformation matrices. This gives the final equation for the propagation of the diabatic expansion coefficients, shown in equation \eqref{eq:DiabPropagation}.
\begin{equation}
  \dot{\mathbf{u}}^{(I)} = \left(-\frac{i}{\hbar} \mathbb{H}^{(I)} - \mathbb{D}^{(I)}_{diab} \right) \mathbf{u}^{(I)} + \left(\mathbb{U}^{(I)} \mathbb{X}_{qm} \mathbb{U}^{(I)})^{-1} \right) \mathbf{u}^{(I)}
  \label{eq:DiabPropagation}
\end{equation}
Where $\mathbb{H}^{(I)}$ is the diabatic Hamiltonian constructed via the AOM method, $\mathbb{D}_{diab}^{(I)}$ are the diabatic nonadiabatic coupling elements ($d_{diab, lk}^{(I)} = \langle \phi_{l} | \dot{\phi}_{k} \rangle$).

\subsection{Forces}
A full derivation of the transformation of basis for the equation propagating forces is given in appendix \ref{app:BasisTrans}. The result is given in equation \eqref{eq:diabForces} below:
\begin{align}
  \begin{split}
	  \mathbf{F}_{eh, \nu}^{(I)} &= \sum_{i,j} \mathbf{u}_{i}^{*(I)} \mathbf{u}_{j}^{(I)} \left( \nabla_{\nu} H_{ij}^{(I)} + \sum_{l} \mathbf{d}_{lk, \nu}^{(I)} H_{lj}^{(I)} - \sum_{l} \mathbf{d}_{lj, \nu}^{(I)} H_{il} \right) \\
	  &- \sum_{l,k} |C_{l}^{(I)}|^2 \left( \sum_{\nu'}^{N_{n}}                \frac{2}{\hbar M_{\nu'}} \mathcal{Q}_{\nu', lk}^{(I)} \cdot                   \mathbf{f}_{l, \nu'}^{(I)} \right)\left[ \mathbf{f}_{k, \nu}^{(I)} -          \mathbf{f}_{l, \nu}^{(I)} \right] |C_{k}^{(I)}|^2
	\end{split}
  \label{eq:diabForces}
\end{align}

\noindent There are a couple of things to note with this equation. Firstly, the quantum momentum part has not been transformed. This is because the forces are basis independent and doesn't need to be transformed. The quantities required for the calculation of this part of the equation are already calculated in order to propagate the coefficients so only a small amount of extra effort is required to calculate the quantum momentum force term. The Ehrenfest part of the equation has been transformed. This is because the nonadiabatic coupling vectors within the adiabatic basis are never required so are never calculated. Further, the commutator term in the diabatic basis has be shown to provide a negligible contribution to the overall force. Seeing as this term requires significant computational effort it can be neglected. This makes the calculation of the Ehrenfest forces in the diabatic basis far cheaper than in the adiabatic basis.

\section{Testing the diabatic propagator}
The diabatic propagation can be tested against the already tested adiabatic propagator using the Tully model Hamiltonian. The code should give the same results, given the same inputs. To check this, in figure \ref{fig:diab_prop_vs_adiab}, the simulations carried out in figure \ref{fig:LitCompCTMQCTully} were repeated though this time the diabatic propagator was used.
\begin{figure}[h]
  \includegraphics[width=\textwidth]{./img/CTMQC/TullyModels/CTMQC_ad_vs_di_wTraj_pops.png}
  \caption{\label{fig:diab_prop_vs_adiab}The 4 Tully models simulated using propagating the equations within a diabatic and adiabatic basis. The green line shows results from the diabatic propagator and the red line shows results from the adiabatic propagator.}
\end{figure}
We can see in figure \ref{fig:diab_prop_vs_adiab} that the results for the adiabatic and diabatic propagator are almost exactly the same for each model. In model 3, where the problem with the divergent $\mathcal{Q}_{lk, \nu}^{(I)}$ doesn't occur, the 2 results are exactly on top of each other. In the other models there is a slight discrepancy. This is due to the unpredictable $\mathcal{Q}_{lk, \nu}^{(I)}$ spikes not being perfectly corrected. However, figure \ref{fig:diab_prop_vs_adiab} still serves as confirmation of the propagation within the diabatic basis.

\section{Simulating Molecular Systems}
To go beyond the 1D Tully model systems the AOM method is combined with CTMQC and applied to an Ethylene dimer. Fortunately, the majority of the code from the Tully model systems can be re-used. In fact, the only difference is the way the Hamiltonian (and diabatic NACE) is constructed. The code for carrying out these tasks (the AOM part) has been implemented by previous members of the group and has been well tested and verified against the literature. Therefore, I will not include any tests of this part of the code in this document but instead refer the reader to the numerous papers discussing AOM and its use in within the fewest switches surface hopping framework \cite{Carof2017FSSH, C9FD00046A, C9CP04770K, FOB-SH_Spencer,        C6FD00107F,FlickPolarons, Giannini2018Crossover, Giannini2019,          C9TC05270D,Gajdos2014, AOM_vs_HigherOrder}. An ethylene dimer was chosen as a reasonable first system due to its relative simplicity (shown in figure \ref{fig:EthDimer}) the total number of atoms is 12 and only 2 electronic states will be considered.
\begin{figure}[h]
  \includegraphics[width=\textwidth]{./img/CTMQC/Ethylene_Annotated.png}
  \caption{\label{fig:EthDimer}An example Ethylene dimer used to test the CTMQC implementation. The right panel shows the positions of just 1 replica. The left panel shows the positions of all replica with the replica shown on the right highlighted in red.}
\end{figure}
The system shown in figure \ref{fig:EthDimer} was initialised in the adiabatic ground state. Positions and velocities were sampled from a short NVT molecular dynamics equilibration. The scaling factor ($C$ in equation \eqref{eq:AOM_proportional}) was chosen to give a coupling of approximately 27 meV. This is approximately $\frac{1}{4} \times$ the reorganisation energy -parameterised to be 100 meV. The amount of charge transfer is dependent on the ratio between reorganisation energy and the electronic couplings ($\frac{H_{ab}}{\lambda}$). The factor of $\frac{1}{4}$ was chosen to be a reasonable factor -seen in other organic semiconducting systems. The nuclear timestep was chosen to be 0.05fs and the electronic one was 0.005 fs. The switch to $\mathbf{R}_{0, \nu}^{(I)}$ was chosen as the correction method of the quantum momentum and 100 trajectories were used.
\\\\
The populations from a trial run using the set up described above was carried out and the populations and norm of the adiabatic expansion coefficients are plotted in figure \ref{fig:CP2K_norm}. In this figure we see large jumps in the norm, these are caused by the divergences in the quantum momentum term. These occur more in this system as it is more complex (more atoms, higher dimensional) and runs for a longer time with more avoided crossings. The fact that there are 12 atoms and 3 Cartesian dimensions instead of 1 means that the $\mathbf{R}_{lk, \nu}^{(I)}$ term must be calculated many more times increasing the probability of happening upon a divergence. The errors can also accumulate meaning that after a few trivial crossings the populations become extremely noisy. This eventually causes the code to crash and results from it cannot be trusted. Most commonly the reason for the code crashing is a large spike in the computed forces caused by a spike in the quantum momentum. This large force then causes the atoms to collide and the code to crash. The code is very stable when just using Ehrenfest dynamics.
\begin{figure}[h]
  \includegraphics[width=\textwidth]{./img/CTMQC/Ethylene_norm.png}
  \caption{\label{fig:CP2K_norm}The norm of the adiabatic expansion coefficients. Thin red lines show the norm for each trajectory and the thick green line shows the average over all trajectories.}
\end{figure}
Many more simulations have been carried out to diagnose and fix this issue. Results from all of these cannot be included in this document though I will provide a brief summary of results below.
\\
\textbf{Varying the number of replicas}
Increasing the number of replicas in the system, somewhat counterintuitively, decreases stability. This is because with more replicas there is more of a chance the code will stumble upon a calculation giving a divergence in the quantum momentum term.
\\
\textbf{Varying the timestep}
Decreasing the timestep does help to improve norm conservation (before the code crashes). However, it does not lead to a more stable simulation that allows for longer timescales to be simulated. This is because decreasing the timestep provides more opportunity for a small numerical error to cause a divergence in the quantum momentum term.
\\
\textbf{Removing Center of Mass Motion}
In some simulations the replicas positions spread out so much that the quantum momentum term became negligible. This was to prevent that from happening.
\\
\textbf{Varying the gaussian width ($\sigma$) parameter}
If the $\sigma$ parameter is set to be large (> 2) then the simulation is more stable as the quantum momentum term is smaller and errors don't accumulate as quickly. However, this the limit of a very large $\sigma$ is Ehrenfest dynamics. If the $\sigma$ is set to be small (< 0.2) the simulation becomes extremely unstable as the quantum momentum forces populations to decohere too quickly. This is discussed in more detail in section \ref{sect:SigmaSect}.
\\
\textbf{Turning off the quantum momentum addition to the force term}
The code runs more stably if the quantum momentum term is not included in the forces. This has also been shown to be much less important for the accuracy of the results than the quantum momentum addition to the coefficients. However, even in this case the code eventually crashes after an accumulation of errors in the coefficients results in erroneous forces resulting in geometries that fail CP2K internal checks.
\\
\textbf{Renormalisation}
This does not seem to help with stability. Furthermore, it merely helps hide the large norm drift and doesn't fix the problems it causes.


varying the number of replicas, varying the coupling strength, varying the timestep, removing center of mass motion, 



\begin{itemize}
	\item Plot Forces (and decomposition of forces -ehren and QM).
	\item Plot Coeffs and Norm
	\item State that the norm goes wild and atoms crash into each other
	\item Relate this back to the quantum momentum term and it's spikes.
	\item Show results of simulation of multiple Tully models
	\item State that this system (even with the correction) still crashed towards the end.
	\item More avoided crossings result in the quantum momentum being used more. Errors build up in these systems causing crashes.
	\item The technique shouldn't be used for systems with multiple avoided crossings. Maybe for systems with just 1 or a few.
\end{itemize}



\section{Conclusions}
The CTMQC method shows great promise as a new nonadiabatic molecular dynamics technique. It was derived as the semi-classical limit of the exact factorisation of the time-dependent electron-nuclear wave function \cite{abedi_exact_2010, agostini_semiclassical_2015}. It purports to handle decoherence corrections in a more rigorous, first principles way without the need of empirical parameters. Although this method was first reported in 2015 \cite{agostini_semiclassical_2015}, there are still very few papers reporting results using this method \cite{min_ab_2017, gossel_coupled-trajectory_2018,agostini_semiclassical_2015}. With the most complex system being restricted to a 7 atom molecule \cite{min_ab_2017}. I believe this is due to the problems with the current formalism.



